\chapter{Conclusions and Future Works}
\label{ch:future}

In the last decades, the number of malware increased significantly. Between all those malware, there are those developed by Advanced Persistent Threat, which are really dangerous. APT is an advanced adversary or a group whose goal is to secretly intrude into computer systems of big companies or foreign governments to steal valuable information. 

In this thesis, we proposed a framework to detect if a suspected malware belongs or not to an APT, so that human analysts can then concentrate their resources to investigate samples tagged as related to an APT with a certain degree of confidence. This thesis enriches the static features used in malware triage \cite{laurenza2017malware}, adding features from code analysis. A classifier is trained with features from disassembled code, control flow graph, cyclomatic complexity, and rich header. Even if static analysis is not sufficient to have a complete knowledge of a malware, it is fast and easy to compute. For an early identification in a malware APT prioritization framework, we have promising results. The precision, and hence the number of false positives, is crucial for this framework. We do not want an analyst focuses on a sample targeted as potential APT when instead is just a common malware. In our thesis, APT samples are identified on average with precision and accuracy over 95\%. Moreover, we identified some APT classes with a precision and accuracy of around 100\%. Overall, using feature selection, we reduced the classification time by 98\%, and improved a little bit the performance, with respect to the original dataset.

However, dAPTaset is not sufficiently large, and we tested only samples known to be related to some APT.
As future works, first of all, we want to test our classification model on a larger dataset. Furthermore, we want to extend this thesis also to non-APT malware. The framework should recognize which samples are non-APT and provide the APT membership of the others. 

Secondly, we want to further investigate some samples that we found as packed. We are interested in writing a script with Ghidra to automatically unpack the packed malware and extract more features to enrich our dataset. Besides, we want to study Ghidra in-depth to take advantage of all its features and develop scripts to automatically extract more information from the binaries.

Moreover, we want to test different classification models and feature selection techniques. We could focus on \textit{Principle Component Analysis} to diminish the dimensionality of our dataset. We could consider deep neural networks as a classification method since many researchers reported successfully results \cite{zhou2018malware} \cite{tobiyama2016malware}. 
\chapter{Conclusions and Future Works}
\label{ch:future}

In the last decades, the number of malware increased exponentially. Between all those malware, there is a particular type known as Advanced Persistent Threat, which is really dangerous. APT is an advanced adversary or a group whose goal is to intrude secretly into computer systems of big companies or foreign governments to steal valuable information. 

We proposed a framework to detect if a suspected malware belongs or not to an APT. The analysts can then concentrate their resources to investigate samples tagged as related to an APT with a certain degree of confidence. This thesis enriches the static features used in malware triage in \cite{laurenza2017malware}, adding features from code analysis. A classifier is trained with features from disassembled code, control flow graph, cyclomatic complexity, and rich header. Even if static features are not sufficient to classify different malware, they are fast and easy to compute. For an early identification in a malware APT prioritization framework, we have promising results. The precision, i.e., the number of false positives, is crucial for this framework. We don't want an analyst to focus on a sample targeted as potential APT when instead is just a random malware. In our thesis, APT samples are identified on average with precision and accuracy over 95\%, some samples, even with a precision ad accuracy of 100\%. Overall, using feature selection, we reduced the classification time, and improved a little bit the performance, with respect to the original dataset.

However, the dAPTaset is not sufficiently large, and we tested only samples known to be related to some APT.
As future works, first of all, we want to test our classification model on a larger dataset. Furthermore, we want to extend this thesis also to non-APT malware. The framework should recognize which samples are non-APT and classify the APT membership of the others. 

Secondly, we want to investigate further some samples that we found as packed. We are interested in writing a script with Ghidra to automatically unpack the packed malware and extract more features to enrich our dataset. Besides, we want to study more Ghidra to take advantage of all its features and develop scripts to extract more information from the binaries automatically.

Moreover, we want to test different classification models and feature selection techniques. We want to focus on \textit{Principle Component Analysis} to diminish the dimensionality of our dataset. We want to consider deep neural networks as a classification method since many researchers reported successfully results \cite{zhou2018malware} \cite{tobiyama2016malware}. 
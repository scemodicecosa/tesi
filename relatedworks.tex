\chapter{Related works}
\section{De-anonymizing Programmers from Executable Binaries}
In this paper, Caliskan et al. presented their approach to de-anonymize different programmers from their compiled programs. They used a dataset of executables from Google Code Jam, and they show that even after compilation the author fingerprints persist in the code, and it is still possible to de-anonymize them.\\

Their approach was to extract distinct blocks of features with different tools and then analyze them to determine the best ones to describe the stylistic fingerprint of the authors precisely. Firstly,  with a disassembler is possible to disassemble the binary and to obtain the low-level features in assembly code.

Then with a decompiler, they extracted the \textbf{Control Flow Graph} and the \textbf{Abstract Syntax Tree}. They determine the stylistics features from those four documents.
\\
In particular, the tools used are \textbf{ndisasm} \textbf{radare2} disassembler for the disassembled code and the Control Flow Graph; \textbf{Hexray} decompiler for the pseudocode, which is passed as input to \textbf{Joern}, a C fuzzy parser, to produce the  \textbf{Abstract Syntax Tree}.\\

They used different types of features selection techniques to reduce the number of features to only 53. They trained a RandomForest Classifier with the dataset created to de-anonymize the authors correctly. \\

This paper is an entry point for our work, and we tried to apply the same approach to the apt triage problem. However, the tools used by Caliskan et al. are outdated and no more maintained, so we decided to use the novel open-source tool ghidra to write the script and extract the information we want. In this way, we significantly reduced the amount of time for feature extraction.

\section{APT triage}
Laurenza et al. show that it is possible to help an analyst lightening the number of samples to analyze. The main idea is to process all the executables, extract some features, and then classify them to determine if they belong or not to a possible APT campaign. The analyst can then analyze only the suspected files that can be related to some APT. Unfortunately, this work has some drawbacks. First of all, it is possible to identify only samples correlated to a known APT campaign, if the sample belongs to a new never investigated APT, then it is impossible to detect it. Furthermore, even if the executable belongs to a known APT, there is no guarantee that the classifier detects it because it just relies on information present in the header of the file. The malware writer can hijack that information to mislead the model. \\


The dataset used by Laurenza et al. is \textbf{dAPTaset}, a public database that collects data related to APTs from existing public sources through a semi-automatic methodology and produces an exhaustive dataset. Unfortunately, the dataset is not big enough and is not perfectly balanced. It contains only 2086 samples because there are not many samples belonging to an APT campaign. Instead, the majority of public analyzed samples are just malware.

\section{Rich Header}
\textbf{da scrivere sunto del lavoro su rich header}
\cite{dubyk2019sans}

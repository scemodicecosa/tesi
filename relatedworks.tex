\chapter{Related works}
\section{APT triage}
Laurenza et al. show that it is possible to help an analyst lightening the number of samples to analyze. The main idea is to process all the executables, extract some features, and then classify them to determine if they belong or not to a possible APT campaign. The analyst can then analyze only the suspected files that can be related to some APT. Unfortunately, this work has some drawbacks. First of all, it is possible to identify only samples correlated to a known APT campaign, if the sample belongs to a new never investigated APT, then it is impossible to detect it. Furthermore, even if the executable belongs to a known APT, there is no guarantee that the classifier detects it because it just relies on information present in the header of the file. The malware writer can hijack that information to mislead the model. \\


The dataset used by Laurenza et al. is \textbf{dAPTaset}, a public database that collects data related to APTs from existing public sources through a semi-automatic methodology and produces an exhaustive dataset. Unfortunately, the dataset is not big enough and is not perfectly balanced. It contains only 2086 samples because there are not many samples belonging to an APT campaign. Instead, the majority of public analyzed samples are just malware.

\section{De-anonymizing Programmers from Executable Binaries}
In this paper, Caliskan et al. presented their approach to de-anonymize different programmers from their compiled programs. They used a dataset of executables from Google Code Jam, and they show that even after compilation the author fingerprints persist in the code, and it is still possible to de-anonymize them.\\

Their approach was to extract distinct blocks of features with different tools and then analyze them to determine the best ones to describe the stylistic fingerprint of the authors precisely. Firstly,  with a disassembler is possible to disassemble the binary and to obtain the low-level features in assembly code.

Then with a decompiler, they extracted the \textbf{Control Flow Graph} and the \textbf{Abstract Syntax Tree}. They determine the stylistics features from those four documents.
\\
In particular, the tools used are \textbf{ndisasm} \textbf{radare2} disassembler for the disassembled code and the Control Flow Graph; \textbf{Hexray} decompiler for the pseudocode, which is passed as input to \textbf{Joern}, a C fuzzy parser, to produce the  \textbf{Abstract Syntax Tree}.\\

They used different types of features selection techniques to reduce the number of features to only 53. They trained a RandomForest Classifier with the dataset created to de-anonymize the authors correctly. \\

This paper is an entry point for our work, and we tried to apply the same approach to the apt triage problem. However, the tools used by Caliskan et al. are outdated and no more maintained, so we decided to use the novel open-source tool ghidra to write the script and extract the information we want. In this way, we significantly reduced the amount of time for feature extraction.



\section{Rich Header}
\textbf{da scrivere sunto del lavoro su rich header}
\cite{dubyk2019sans}

\section{Advanced Persistent Threat}

APT stands for Advanced Persistent Threat, a kind of sophisticated attack which requires an advanced level of expertise and aims to remain persistent on the attacked infrastructure.

The term APT can refer to a persistent attack with a specific target, or it can refer to the group that organized the attack, sometimes the group is affiliated with some sovereign state.
\\

To understand better what is an APT, we need to decompose the word: 

\textbf{Advanced:} the people behind the attack have an advanced level of expertise, resources, and money. They usually do not use known malware, but they write their malware specific to the target they want. Moreover, they can gather information on the target from the intelligence of their country of origin.

\textbf{Persistent:}  The adversary does not aim to gain access in the most number of system, but rather to have persistent access to the infrastructure. The more time they remain undiscovered in the organization's network infrastructure, the higher are the chances of lateral movement, the greater are the information they can gather. Persistent access is the key to every APT.

\textbf{Threat:} As said before, this is an organized threat, with a strategical vision of what to achieve. It is not an automatic tool that attacks everything trying to gather something. It is a meticulously planned attack that aims to obtain certain information from a given organization. \cite{apt_def}
\\

In general, APTs aim to higher-value targets like other nations or some big corporations. However, any individual can be a target. FireEye publish a report each year about the new APT campaign, the diagram below states which industry is the most attacked in the last year.\\

\begin{figure}[!h]
	\centering
	\includegraphics[width=1.0\columnwidth]{graph}
	\caption{Diagram of industry target}
\end{figure}

A point of particular concern is the retargeting, in the Americas, 63\% of the companies attacked by an APT, are attacked again last year by the same or similar group. In the Asian and Pacific areas, this is even worse, 78\% of the industries are hacked again. \cite{fireeye_mtrends} \\


\begin{figure}[ht!]
	\centering
	\includegraphics[width=0.5\columnwidth]{retarget}
	\caption{Retargeting divided by regions}
\end{figure}

Advanced persistent threats, contrary to regular malware, are composed of different phases, each of which has an important role. 

The attack is decomposed into smaller steps, for example, if a group of hackers wants to attack a CEO of a given company, they will not send directly to the CEO a phishing email, because it's likely that he has a complex system of security and they would be detected instantly. 

Instead, the first step would hack a person in the same company with lower permissions that can have minor defense mechanisms. Once they got the first computer, they can explore the network infrastructure of the organization, and then decide which action is the best.
They could cover their track from the log system, or locate the data they need or send a phishing email to the CEO from the owned user.\\

So how does an APT work? Fireeye described their behavior in six steps. \cite{fireeye_anatomy}


\begin{enumerate}
	\item The adversary gains access into the network infrastructure, installing a malware sent through a phishing email or by exploiting some vulnerability.
	\item Once they comprised the network, the malware scans all the infrastructure looking for other entry points or weaknesses. It can communicate with a Command \& Control server (C\&C) to receive new instructions or to send information.
	\item The malware typically establishes additional points of compromise to ensure that the attack can continue even if a position is closed.
	\item Once the attackers have a reliable connection to the network, they start dumping data such as usernames and passwords, to gain credentials.
	
	\item The malware sends the data to a server where the attackers can receive the information. Now the network is breached.
	
	\item The malware tries to cover its tracks cleaning the log system, but the network is still compromised so the adversary can enter again if they are not detected.
	
\end{enumerate}
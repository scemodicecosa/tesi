% !TeX encoding = UTF-8
% !TeX program = pdflatex
% !TeX spellcheck = en_US
\documentclass[binding=0.6cm,LaM,oneside]{sapthesis} % LaM for a Laurea Magistrale
\usepackage{microtype}
\usepackage[utf8]{inputenc}
\usepackage{hyperref}
\usepackage{wrapfig}
\usepackage{float}
\usepackage{booktabs}
\usepackage{amsmath}
\usepackage{algorithm}
\usepackage[noend]{algpseudocode}

\makeatletter
\def\BState{\State\hskip-\ALG@thistlm}
\makeatother

\usepackage{lineno}
\modulolinenumbers[5]
\usepackage{graphicx}
\usepackage{booktabs}
\usepackage{amssymb,amsmath,nccmath}
\usepackage{cclicenses}
\usepackage{makecell}
\usepackage{lscape,array}
\newcolumntype{C}[1]{>{\centering\arraybackslash}p{#1}} 
\usepackage[thin, , thinc]{esdiff}
\usepackage{subcaption}
\usepackage{caption}
\usepackage{framed}  
\usepackage{subcaption}
\usepackage[font=small,skip=0pt]{caption}

\PassOptionsToPackage{hyphens}{url}
\hypersetup{
	colorlinks = true, % Colours links instead of ugly boxes
	urlcolor = black, % Colour for external hyperlinks
	linkcolor = black, % Colour of internal links
	citecolor = black % Colour of citations
}

\usepackage{listings}
\usepackage{xcolor}
\usepackage{multirow}


\renewcommand\baselinestretch{1.5}
\colorlet{punct}{red!60!black}
\definecolor{background}{HTML}{EEEEEE}
\definecolor{delim}{RGB}{20,105,176}
\colorlet{numb}{magenta!60!black}

\lstdefinelanguage{json}{
	basicstyle=\normalfont\ttfamily,
	numbers=left,
	numberstyle=\scriptsize,
	stepnumber=1,
	numbersep=8pt,
	showstringspaces=false,
	breaklines=true,
	frame=lines,
	backgroundcolor=\color{background},
	literate=
	*{0}{{{\color{numb}0}}}{1}
	{1}{{{\color{numb}1}}}{1}
	{2}{{{\color{numb}2}}}{1}
	{3}{{{\color{numb}3}}}{1}
	{4}{{{\color{numb}4}}}{1}
	{5}{{{\color{numb}5}}}{1}
	{6}{{{\color{numb}6}}}{1}
	{7}{{{\color{numb}7}}}{1}
	{8}{{{\color{numb}8}}}{1}
	{9}{{{\color{numb}9}}}{1}
	{:}{{{\color{punct}{:}}}}{1}
	{,}{{{\color{punct}{,}}}}{1}
	{\{}{{{\color{delim}{\{}}}}{1}
	{\}}{{{\color{delim}{\}}}}}{1}
	{[}{{{\color{delim}{[}}}}{1}
	{]}{{{\color{delim}{]}}}}{1},
}

\hypersetup{pdftitle={Investigation of static features for improved APT malware identification},pdfauthor={Leonardo Sagratella}}
\title{Investigation of static features for improved APT malware identification}
\author{Leonardo Sagratella}
\IDnumber{1645347}
\course{CyberSecurity}
\courseorganizer{Facoltà di Ingegneria dell'informazione, informatica e statistica}
\AcademicYear{2019/2020}
\copyyear{2020}
\advisor{Prof. Riccardo Lazzeretti}
\coadvisor{Dr. Giuseppe Laurenza}
\authoremail{leonardosagratella@gmail.com}
\begin{document}
	\frontmatter
	\maketitle
	\dedication{
		Dedicato alla mia famiglia, mio padre, mia madre e mia sorella che mi sono sempre stati vicini e mi hanno sempre sostenuto in questo percorso. Ai miei compagni di corso conosciuti in questi anni, in particolare Davide, Gianluca e Gianmarco grazie davvero. Dedicato ai miei piu grandi amici di sempre del campeggio Antonio, Cecilia, Federico e Lucia, ma in particolare a quel matto in culo di Federico.
		Il rigraziamento più grande va alla mia ragazza e compagna di corso Federica, che mi ha sempre supportato e sopportato in questi anni.
}
 	\begin{abstract}
		In the last years, a new cyber-threat is spreading fast, known as \textit{Advanced Persistent Threat} (APT). An APT is a group that aims to penetrate and exflitrate information from high-valuable targets. They rely on malware and tools developed for the specific target they want to attack, usually governments, critical infrastructures or big organizations. Their goal is to break into the target's systems and stay undetected as long as possible. Meanwhile, they steal valuable information and try new ways to infect other target's computers. We propose a framework for malware prioritization to detect, among all the new malware samples that every day researchers find, the one that can be correlated to an APT group. The objective of this thesis is to correctly identify samples related to known APT. Those samples are then dispatched to analysts for further and manual inspection. To make this process as fast as possible we rely only on static features, without the need of executing the sample. We applied different techniques of feature selection to reduce the dataset dimensionality. We trained a \textit{RandomForestClassifier} with the selected features and we analyze its performances. In particular, we focus on precision because we do not want the analysts to waste time on false positives. Overall, we obtained great performances on both precision and accuracy, while runtime has been significantly reduced respect to the state-of-the-art.
	\end{abstract}
	\tableofcontents
	\mainmatter
	\chapter{Introduction}

In the last decades, we are witnessing the constant improvement of technology and its utility in our everyday life. As technology advances, the cyber-threat arises.
Recentl,y the number of malware rapidly increased. In the last quarter of 2017, McAfee discovered 63.4 million new malware \cite{mcafee2018}, the highest of all time. In 2018 and 2019 AvLab reported on average 12.05 million new samples per month \cite{avtest2020}. Among those malware, there is a small quantity developed by Advanced Persistent Threats (\textbf{APT}), that is threatening and continuously increasing. 

The National Institute of Standards and Technology (\textbf{NIST}) defines the \textbf{Advanced Persistent Threat} as \cite{nistapt} \textit{"An adversary that possesses sophisticated levels of expertise and significant resources which allow it to create opportunities to achieve its objectives by using multiple attack vectors (e.g., cyber, physical, and deception). These objectives typically include establishing and extending footholds within the information technology infrastructure of the targeted organizations for purposes of exfiltrating information, undermining or impeding critical aspects of a mission, program, or organization; or positioning itself to carry out these objectives in the future."}

The term APT refers to an advanced adversary or a group, that aims to attack some critical infrastructures: a government, or an organization. They use unique attack vectors and malware ad-hoc developed for a specific target, to attack systems and exfiltrate valuable information.

To better understand what is an APT, we need to decompose the word: \cite{apt_def}

\textbf{Advanced:} the people behind the attack have an advanced level of expertise, resources, and money. They usually do not use known malware, but they write their malware specific to the target they want. Moreover, they can gather information on the target from the intelligence of their country of origin.

\textbf{Persistent:}  The adversary does not aim to gain access in the most number of system, but rather to have persistent access to the infrastructure. The more time they remain undiscovered in the organization's network infrastructure, the higher are the chances of lateral movement, the greater are the information they can gather. Persistent access is the key to every APT.

\textbf{Threat:} As said before, this is an organized threat, with a strategical vision of what to achieve. It is not an automatic tool that attacks everything trying to gather something. It carries on meticulously planned attack, that aim to obtain certain information from a given organization. 

In general, APTs aim to higher-value targets like other nations or some big corporations. However, any individual can be a target. FireEye publish a report each year about the new APT campaigns. Figure \ref{fig:diag} states the main targets in the last year.\\

\begin{figure}[!h]
	\centering
	\includegraphics[width=1.0\columnwidth]{graph}
	\caption{Diagram of APT targets}
	\label{fig:diag}
\end{figure}


%A point of particular concern is the retargeting, in the Americas, 63\% of the companies attacked by an APT, are attacked again last year by the same or similar group. In the Asian and Pacific areas, this is even worse, 78\% of the industries are hacked again. \cite{fireeye_mtrends} \\


%\begin{figure}[ht!]
%	\centering
%	\includegraphics[width=0.5\columnwidth]{retarget}
%	\caption{Retargeting divided by regions}
%\end{figure}

\textit{Advanced persistent threats}, contrary to regular malware, are composed of different phases, each one having an important role. 

The attack is decomposed into smaller steps, for example, if a group of hackers wants to attack a CEO of a given company, they will not send directly to the CEO a phishing email, because he likely has a complex system of security and they would be detected instantly. 
Instead, the first step would hack a person in the same company with lower permissions that can have minor defense mechanisms, or hack some organization that works with the one they want to attack. Once they are established the first computer, they can explore the network infrastructure of the organization, and then decide which action is the next one.
They could cover their track from the log system, locate the data they need, or send a phishing email to the CEO from the owned user.

So how does an APT work? Fireeye described their behavior in six steps \cite{fireeye_anatomy}.

\begin{enumerate}
	\item The adversary gains access into the network infrastructure, installing a malware sent through a phishing email or by exploiting some vulnerability.
	\item Once they comprised the network, the malware scans all the infrastructure looking for other entry points or weaknesses. It can communicate with a Command \& Control server (C\&C) to receive new instructions or to send information.
	\item The malware typically establishes additional points of compromise to ensure that the attack can continue even if a position is closed.
	\item Once the attackers have a reliable connection to the network, they start dumping data such as usernames and passwords, to gain credentials.
	
	\item The malware sends the data to a server where the attackers can receive the information. Now the network is breached.
	
	\item The malware tries to cover its tracks cleaning the log system, but the network is still compromised so the adversary can enter again if they are not detected.
\end{enumerate}

Since APT targets critical infrastructures, governments, organizations, they are mighty and critical. Analysts and researchers work every day to dissect and analyze malware. However, due to their constant increase, researchers can no more handle all of them. 
We proposed a prioritization system to dispatch to analysts only the samples that can be related to an APT campaign. Our system extracts static features from malware samples, to provide a fast and efficient mechanism for APT detection. The main goal of this thesis is to correctly identify samples belonging to the same APT group. Starting from the malware triage \cite{laurenza2017malware} idea presented by Laurenza et al. we tried a different approach using different features from code analysis. Caliskan et al. \cite{caliskan2015anonymizing} inspired us with their technique for de-anonymizing programmers from their executable. We apply the same concept to the malware triage using different tools, and we manage to correctly identify APT classes with promising results. 

We use Ghidra as reverse engineering software to automate the features extraction phase. We rely on Ghidra's scripting capability to automatically extract different information for creating our dataset. Furthermore, we enrich our dataset using the Rich Header information as explained by Dubik\cite{dubyk2019sans}.

We propose different techniques for feature selection and compare their results. In the end, we reduce the number of features by 99.85\% with respect to the original dataset, and we reduce the classification time approximately by 98\%. We obtain great results in both accuracy, precision, recall and f1-score, on average around 95\%. Precision is our main concern since we are developing a prioritization framework, we do not want to dispatch a false postive to human analysts. Some APT classes are detected with a precision of 100\%.


The thesis is organized as follows.
In Chapter \ref{ch:rel-works}, we analyze all the paper that we use to start this thesis. We talk about a novel framework for APT malware prioritization proposed by Laurenza et al. In the next section, we analyze the work made by Caliskan et al. \cite{caliskan2015anonymizing}, where they show how it is still possible to de-anonymize different programmers just relying on executable files. The last work of Dubyk \cite{dubyk2019sans} shows the usage of an undocumented section of the PE header, Rich Header, in malware classification.

Chapter \ref{ch:prelim} introduces the preliminary notions needed for the comprehension of this thesis. It focuses on Reverse Engineering (\textbf{RE}) and shows which documents are useful for an analyst in code analysis. Then we present the state-of-the-art of RE tools, the reasons behind our choice on the tool Ghidra, and all its functionalities useful for code analysis. Furthermore, we present the tools used for machine-learning tasks.

Chapter \ref{ch:feat-cre} focuses on features extraction from the dataset using Ghidra. We shows the different block of features proposed, and for each of them, we present the type of features selected and the extraction process with Ghidra.

Chapter \ref{ch:classif} introduces all the machine-learning models and techniques used. It presents the \textit{RandomForest} and \textit{XGBoost} classifier, and the problem of validating results obtained from a model. Moreover, we present the different techniques for feature selection used to shrink our feature vector.

Chapter \ref{ch:disc} comprehends all the tests we made to evaluate our models. We present in detail the results of feature selection and how we chose the best for our purpose.
Furthermore, we analyze the performances of the two models.

In Chapter \ref{ch:future} we show the future works we could make to improve this thesis. 
	\chapter{Related works}

This chapter presents the related works we started from to develop this work. Firstly we introduce a paper from Laurenza et al. that explain how to build a malware triage for APT classification. The second work, from Caliskan et al., shows how is it possible to de-anonymize different programmers even with a compiled executable. The last work of Dubyk propose an approach for malware similarities based on the Rich Header.

\section{Malware Triage Based on Static Features and Public APT Reports}
Laurenza et al. show that it is possible to help an analyst lightening the number of samples to analyze. They proposed an architecture for sample prioritization, which, based on a rank score, decides which sample has the priority to be further inspected by an analyst. In this way, it is possible to avoid wasting time analyzing samples that are not important. 

They decide to rely only on static analysis instead of dynamic analysis to improve the speed of processing. In this architecture, the time elapsed for analyzing a classifying the samples is more important than the accuracy of the classification. Since it relies on static analysis, there is no need for sample execution, or to set up a complex virtualized environment. Every sample is processed, and some features are extracted from the header. The knowledge base created is used to classify new malware.

\begin{figure}
	\centering
	\begin{subfigure}{1.0\textwidth}
		\centering		\includegraphics[width=1.0\linewidth]{train.png}  
		\caption{Training phase}
		\label{fig:sub-train}
	\end{subfigure}
	
	\begin{subfigure}{1.0\textwidth}
		\centering
		% include second image
		\includegraphics[width=1.0\linewidth]{test.png}  
		\caption{Testing phase}
		\label{fig:sub-test}
	\end{subfigure}
	
	\caption{Malware triage flow.}
	\label{fig:mal_triage}
\end{figure}

They proposed a malware triage architecture, based on the identification of malware similar to other malware known to be related to some APTs campaign. The idea is to collect public APT reports and their related samples.  Then, extracting features from the samples and assigning them to the class of the APT they belong to. Those features vectors are used to train a classifier.



Figure \ref{fig:mal_triage} shows the architecture of both the training and testing phases. The APTs loading plugin frequently loads the information on APT crawled from public reports on the internet. The loading plugin feeds the system with new samples from different sources. The features extraction phase produces the features vectors from the malware, continuously updated by the previous loading plugin. In the classifier phase, a classifier is trained with features vector. 

In the analysis phase, novel samples follow the same pipeline; the features extraction stage extracts the features vector that is passed to the classifier. If the output of the classifier is positive, i.e., the sample is related a know APT, then the sample follows a different path, and is sent a human analyst for further inspection, otherwise it is discarded.

Unfortunately, this work has some drawbacks. First of all, it is possible to identify only samples correlated to a known APT campaign, if the sample belongs to a new never investigated APT, then it is impossible to detect it. Furthermore, even if the executable belongs to a known APT, there is no guarantee that the classifier detects it because it just relies on information present in the header of the file. The malware writer can hijack that information to mislead the model.

\textbf{BOOOO}
\\

The idea in our work is to boost up Laurenza's et al. work, introducing different features without slowing down the entire flow. As written in their future works, they wanted to extend the architecture to new kind of features, based on code analysis. 





The dataset used by Laurenza et al. is \textbf{dAPTaset}, a public database that collects data related to APTs from existing public sources through a semi-automatic methodology and produces an exhaustive dataset. Unfortunately, the dataset is not big enough and is not perfectly balanced. It contains only 2086 samples because there are not many samples belonging to an APT campaign. Instead, the majority of public analyzed samples are just malware.

\section{De-anonymizing Programmers from Executable Binaries}
In this paper, Caliskan et al. presented their approach to de-anonymize different programmers from their compiled programs. They used a dataset of executables from Google Code Jam, and they show that even after compilation the author fingerprints persist in the code, and it is still possible to de-anonymize them.\\

Their approach was to extract distinct blocks of features with different tools and then analyze them to determine the best ones to describe the stylistic fingerprint of the authors precisely. Firstly,  with a disassembler is possible to disassemble the binary and to obtain the low-level features in assembly code.

Then with a decompiler, they extracted the \textbf{Control Flow Graph} and the \textbf{Abstract Syntax Tree}. They determine the stylistics features from those four documents.
\\
In particular, the tools used are \textbf{ndisasm} \textbf{radare2} disassembler for the disassembled code and the Control Flow Graph; \textbf{Hexray} decompiler for the pseudocode, which is passed as input to \textbf{Joern}, a C fuzzy parser, to produce the  \textbf{Abstract Syntax Tree}.\\

They used different types of features selection techniques to reduce the number of features to only 53. They trained a RandomForest Classifier with the dataset created to de-anonymize the authors correctly. \\

This paper is an entry point for our work, and we tried to apply the same approach to the apt triage problem. However, the tools used by Caliskan et al. are outdated and no more maintained, so we decided to use the novel open-source tool ghidra to write the script and extract the information we want. In this way, we significantly reduced the amount of time for feature extraction.



\section{Rich Header}

The work of Maksim Dubyk \cite{dubyk2019sans} shows how the rich header produces a robust series of data points that can be exploited in static PE-based malware detection. The Rich Header is an undocumented section of \textit{Portable Executable} (\textbf{PE}) header and provides a view of the environment where the executable was built. Firstly, Dubik shows the location of the Rich Header, how to understand and decrypt it, and how to extract it. Then, he shows different techniques to leverage the header for PE-based malware classification. The Rich Header is part of our features vector.

When building a PE, there are two distinct phases. The first one is the \textit{compilation phase}, where the high-level instructions from programming language are compiled into machine code, executable by the computer. The second one is the \textit{linking phase},a combination of those different machine code objects into a single executable. The act of combining different objects creates the Rich Header. However, only executables compiled with Microsoft Linker contain the rich header, other compilers, such as gcc or .Net,  do not have it.

\begin{figure}[!h]
	\centering
	\includegraphics[width=1.0\columnwidth]{cmd.png}
	\caption{Hexview of cmd.exe}
	\label{fig:cmd}
\end{figure}

Analyzing an executable with a hexdump, the Rich Header starts from location \texttt{0x80} until the bytes sequence \texttt{0x52696368}, which means \textbf{"Rich"} in ASCII. Figure \ref{fig:cmd} shows the location of Rich header and the ending string "Rich". The content of the rich header is encrypted, and the corresponding decryption key and checksum are located right after the "Rich" string. 

The algorithm that computes the decryption key takes advantage of two checksum algorithms. The first checksum is generated from the bytes that make the DOS header, but with the \texttt{e\_lfanew} field zeroed out\cite{richHeaderHunting}. The second checksum is the combination of each value in the array generated. The outputs of the checksum algorithms are summed together and then masked with the bytes \texttt{0xFFFFFFF}, the key generated is then used to encrypt the rich header section using the XOR.

It is straightforward to decrypt the rich header section. The key found after the "Rich" string is XORed backward with the chunks of the section until the end string "DanS" is found.  

The decrypted rich header is an array that stores metadata related to each phase of the linking process. Each array's element is an 8-byte structure that contains three fields: \textit{product identification} (\textbf{pID}), \textit{product version} (\textbf{pV}), and a \textit{product count} (\textbf{pC}). Those numbers identify the Microsoft product used in that linking phase. However, since the Rich Header is an undocumented section, there is no official mapping of pID with Microsoft products. Nevertheless, some researches partially mapped out some pID with known compilers \cite{richGit}.

To calculate the rich header of each sample, we used the python script provided in \cite{dubyk2019sans}.


\section{Advanced Persistent Threat}

APT stands for Advanced Persistent Threat, a kind of sophisticated attack which requires an advanced level of expertise and aims to remain persistent on the attacked infrastructure.

The term APT can refer to a persistent attack with a specific target, or it can refer to the group that organized the attack, sometimes the group is affiliated with some sovereign state.
\\

To understand better what is an APT, we need to decompose the word: 

\textbf{Advanced:} the people behind the attack have an advanced level of expertise, resources, and money. They usually do not use known malware, but they write their malware specific to the target they want. Moreover, they can gather information on the target from the intelligence of their country of origin.

\textbf{Persistent:}  The adversary does not aim to gain access in the most number of system, but rather to have persistent access to the infrastructure. The more time they remain undiscovered in the organization's network infrastructure, the higher are the chances of lateral movement, the greater are the information they can gather. Persistent access is the key to every APT.

\textbf{Threat:} As said before, this is an organized threat, with a strategical vision of what to achieve. It is not an automatic tool that attacks everything trying to gather something. It is a meticulously planned attack that aims to obtain certain information from a given organization. \cite{apt_def}
\\

In general, APTs aim to higher-value targets like other nations or some big corporations. However, any individual can be a target. FireEye publish a report each year about the new APT campaign, the diagram below states which industry is the most attacked in the last year.\\

\begin{figure}[!h]
	\centering
	\includegraphics[width=1.0\columnwidth]{graph}
	\caption{Diagram of industry target}
\end{figure}

A point of particular concern is the retargeting, in the Americas, 63\% of the companies attacked by an APT, are attacked again last year by the same or similar group. In the Asian and Pacific areas, this is even worse, 78\% of the industries are hacked again. \cite{fireeye_mtrends} \\


\begin{figure}[ht!]
	\centering
	\includegraphics[width=0.5\columnwidth]{retarget}
	\caption{Retargeting divided by regions}
\end{figure}

Advanced persistent threats, contrary to regular malware, are composed of different phases, each of which has an important role. 

The attack is decomposed into smaller steps, for example, if a group of hackers wants to attack a CEO of a given company, they will not send directly to the CEO a phishing email, because it's likely that he has a complex system of security and they would be detected instantly. 

Instead, the first step would hack a person in the same company with lower permissions that can have minor defense mechanisms. Once they got the first computer, they can explore the network infrastructure of the organization, and then decide which action is the best.
They could cover their track from the log system, or locate the data they need or send a phishing email to the CEO from the owned user.\\

So how does an APT work? Fireeye described their behavior in six steps. \cite{fireeye_anatomy}


\begin{enumerate}
	\item The adversary gains access into the network infrastructure, installing a malware sent through a phishing email or by exploiting some vulnerability.
	\item Once they comprised the network, the malware scans all the infrastructure looking for other entry points or weaknesses. It can communicate with a Command \& Control server (C\&C) to receive new instructions or to send information.
	\item The malware typically establishes additional points of compromise to ensure that the attack can continue even if a position is closed.
	\item Once the attackers have a reliable connection to the network, they start dumping data such as usernames and passwords, to gain credentials.
	
	\item The malware sends the data to a server where the attackers can receive the information. Now the network is breached.
	
	\item The malware tries to cover its tracks cleaning the log system, but the network is still compromised so the adversary can enter again if they are not detected.
	
\end{enumerate}
	\chapter{Preliminaries}
\textbf{forse da ampliare disassembler e decompiler}\\
This chapter focuses on the information needed to understand this thesis and the choice we made. We firstly introduce reverse engineering, with all its components. Then we present the state-of-the-art in reverse engineering tools, the reasons we choose Ghidra, and its features. Lastly, we discuss scikit-learn and Jupyter notebook, both used in machine learning tasks.

\section{Reverse Engineering}

Reverse engineering is the process of decomposing a human-made object to understand the underlying architecture, how it works, or to extract some information from it. This process can be applied in various fields, such as computer science, electronic, mechanical, or chemical \cite{eilam2011reversing}.

We focus on reverse engineering applied to computer science. 
\\

The Institute of Electrical and Electronics Engineers (\textbf{IEEE}) states that reverse engineering is \textit{"The process of analyzing a subject system to identify the system's components and their interrelationships, and to create representations of the system in another form or at a higher level of abstraction."}, where the \textit{"subject system"} is referred to the software development \cite{chikofsky1990reverse}. 

When somebody writes a software, he writes it with a language that is understandable by a human, for example, C or Java.  But the computer cannot read it, so the programmer needs to compile the source code to let the computer understand what the software should do. 

The compiling process is the process of translating the source code into a language understandable by a computer. But once the program is compiled, a human can no more read it, unless it has the corresponding source code. 

If we want to understand a binary executable, but its source code is not available, we need to reverse it. There are different techniques of reversing for binary executable: disassembling the binary using a disassembler, decompiling the binary with a decompiler, or analyzing the information exchanged with a bus sniffer or a packet sniffer.


\subsection{Disassembly code}

Assembly language is a low-level programming language that has a substantial correspondence with the architecture's machine code. The program used to convert the assembly instruction to machine code instruction is called assembler. Since the assembly depends on the machine code, there is an assembly language, with its assembler, for each architecture.

With a disassembler, it is possible to revert the actions made by the assembler, so it's possible to translate machine code instruction to assembly language. The output code is formatted for human readability.

\subsection{Decompiler?}

The decompiler is a program that takes as input a binary executable and produces as output a high-level representation of the source code of the program. It is the opposite of a compiler, which, given a source code, generates an executable. The output of the decompiler can be recompiled, and the executable will have the same behavior as the first one. 

Unfortunately, the decompiler is not able to revert correctly the executable, and often it produces obfuscated code. The obfuscated code behaves in the same way, but it is harder for an analyst to understand it.

\subsection{Control Flow Graph}

The Control Flow Graph is a representation, in graph format, of the execution flow of a program or application. Frances E. Allen proposed it in \cite{allen1970control}.

The Control Flow Graph (\textbf{CFG}) is a directed graph, and it is process-oriented. Each node represents a basic block, a sequence of instructions that are executed consecutively without any jump. The edges of the graph represent the path of the execution. The graph represents all the possible paths that the program can take.

There are two types of blocks:\textit{ entry blocks} and \textit{exit blocks}.
The \textit{entry blocks} are the ones where the flow starts; the \textit{exit blocks} are the ones where the flow ends. Figure \ref{fig:cfg} represents some examples of Control Flow Graph of different statements and loop. 

The CFG is useful in code analysis, to determine if some portion of the code is inaccessible.

\begin{figure}[!h]
	\centering
	\includegraphics[width=1.0\columnwidth]{cfg.png}
	\caption{CFG of different statements and loops. \\
	(a) if-then-else\\
	(b) while loop\\
	(c) natural loop with two exit points\\
	(d) loop with two entry points	
}
	\label{fig:cfg}
\end{figure}

\subsection{Cyclomatic Complexity}
Cyclomatic complexity is a software metric to determine the complexity of a single function, a module, a method, some classes, or the entire program.

It is measured starting from a Control Flow Graph of the program and indicates the number of linearly independent paths of it. 
\\
The formula to calculate the complexity is:

\texttt{M =  E - N + 2P},\\
where \texttt{E} indicates the number of \textit{edges}, \texttt{N} specifies the number of \textit{nodes}, and \texttt{P} indicates the number of \textit{connected components}.\\

If the entry point and exit point are connected, we have a strongly connected graph, and the formula for complexity is slightly different: 

\texttt{M = E - N + P}

\begin{algorithm}
	\caption{Example of function with different loops and statements}\label{alg:complexity}
	\begin{algorithmic}[1]
		
		\setcounter{ALG@line}{0}
		\While{not \texttt{EOF}}
		\State{\texttt{Read} $record$}
		\setcounter{ALG@line}{1}
		\If{$field1 = 0$}
		\State{$total \gets total + field1$}
		\setcounter{ALG@line}{2}
		\State{$counter \gets counter + 1$}
		\Else{ }
		\setcounter{ALG@line}{3}
		\If{$field2 = 0$}
		\State{\texttt{Print} $counter$, $total$}
		\setcounter{ALG@line}{4}
		\State{$counter \gets 0$}
		\Else{}
		\setcounter{ALG@line}{5}
		\State{$total \gets total - field2$}
		
		\EndIf
		\State{\textbf{end if}}
		
		\EndIf
		\State{\textbf{end if}}
		\setcounter{ALG@line}{7}
		\State{\texttt{Print} \textit{End record}}
		
		
		
		\EndWhile
		
		\State{\texttt{Print} $counter$}
		
		
		\setcounter{ALG@line}{0}
		
		
	\end{algorithmic}
\end{algorithm}

Algorithm \ref{alg:complexity} shows an example of function with different loops and statements,which corresponding Control Flow Graph is given in Figure \ref{fig:complexity_ex}. The numbers before the lines represent the id of the basic block to which they belong, i.e. the graph's nodes.

Being \textbf{node \#1} the entry node, and \textbf{node \#9} the exit code, we can manually calculate the number of independent path of the function:
\begin{itemize}
	\item 1, 9
	\item 1, 2, 3, 8, 1, 9
	\item 1, 2, 4, 5, 7, 8, 1, 9
	\item 1, 2, 4, 6, 7, 8, 1, 9
\end{itemize}

	Using the formula presented above, it is possible to calculate the complexity of the function that is equal to 4. $M = 11 - 9 + 2 \times 1 = 4$, where 11 is the \textit{number of edges}, 9 the \textit{number of nodes}, and $2 \times 1$ the \textit{number of connected components}. The complexity calculated equals the number of independent paths of the graph.

\begin{figure}[!h]
	\centering
	\includegraphics[width=0.6\columnwidth]{complexity.png}
	\caption{Control Flow Graph of Algorithm \ref{alg:complexity}}
	\label{fig:complexity_ex}
\end{figure}


\section{Reverse Engineering tools}

There are a lot of tools in the market that helps in reverse engineering. Some of them have tons of functionalities, and others can do just a few things.

The most important is \textbf{IDA}, a reverse engineering software developed by Hex-Rey that can achieve different things. It is available in a free version and a pro version. 

It is compatible with most of the executable from different OSes, such as \texttt{PE}, \texttt{ELF}, \texttt{Mach-O}, or even raw binaries. It has extensive support and compatibility with almost every family processors.

The free version contains all the features necessary for some basic reverse engineering; it has a disassembler and performs an automatic analysis of the sample, determining the API used, which parameters are passed to them, and other information. 

The analyst can navigate the code and add some notation, rename functions, and variables, to better understand the behavior of the binary. However, most of the functionalities works only with the PRO version, which costs 9000\$.

The main features of IDA PRO \cite{ida} are the debugger that lets the analyst debug the executables, the possibility of writing scripts to run against the sample, and a disassembler to revert the binary into some source code. 


Another famous framework for RE is \textbf{radare2} \cite{radare2}. It is free and offers a decompiler and a disassembler.  It is compatible with tons of processors, and executable types. It comes with a set of command-line utilities that can be used individually or together, but it also has a graphic interface to navigate the code and the possibility of running scripts. However, it is not user-friendly as other tools .

\textbf{Ghidra} \cite{ghidra} is a novel open-source reverse engineering framework developed by the National Security Agency (NSA). It facilitates the analysis of malicious code and malware like viruses and can give cybersecurity professionalists a better understanding of potential vulnerabilities in their networks and systems. It is a direct competitor of IDA PRO because it offers almost all the features that IDA PRO has, but Ghidra is entirely free. It is possible to run a headless version of Ghidra, and control it by remote. We installed it on a virtual machine on a server and use it to run headless scripts on the executables to extract information about the binaries.

We chose to run our tests on \textbf{Ghidra} for different reasons. First of all, it is entirely free and open-source.
Secondly, it has an headless version of the framework. The possibility of a headless version was perfect for our needs because we can let it run on a server, always powered on, without bothering about resource consumption. 
Moreover, we wanted to simplify as much as possible the extraction of features, and we do not want merging results from different tools, so we decided to use only one software. 
Last but not least, the scripting part of Ghidra was perfect for running analysis and extracting information from the samples. 

The only inconvenient is that, since it was released less than a year ago, the documentation, the support, and the community are not well developed, so at first was hard understanding how the software works.
\section{Ghidra}

This section covers the Ghidra's features regarding the disassembled code, decompiler, Control Flow Graph and complexity presented above.
 
\begin{figure}[ht!]
	\centering
	\includegraphics[width=0.35\textwidth]{ghidra}
	
\end{figure}

\subsection{Disassembled code}

The extraction of disassembled code is an easy and fast task. The documentation provides all the information on how to correctly use the disassembler. 

Ghidra does not have a functionality to extract all the disassembled code, like in the GUI. But it is possible, for each function, to iterate the instructions in a given address space range. The instruction object has a \texttt{toString()} method that returns the disassembled line, that we use to create the features needed.

\subsection{PCode}
PCode is a register transfer language developed by NSA for the reverse engineering framework Ghidra. The idea behind PCode is to create a language as general as possible, to let it adapt to as many different processors architecture. An intermediate language that can model the behavior of different processors is fundamental to develop a comprehensive reverse engineering framework.

The idea behind PCode is that each processor instruction can be expressed as a sequence of PCode. Developers translated each instruction into a sequence of PCode operation as input and output variables (\textbf{varnodes}).

The entire set of single PCode instruction comprises a set of arithmetic and logic instructions that almost all processors perform. The direct translation of the instructions into those operations is called raw-PCode, which can be used to emulate a single processor instruction directly. NSA developed the PCodes to facilitate the construction and the analysis of a data flow graph of disassembled instructions.

A PCode operation is the analog of a machine instruction.  All PCode operations have the same format; they have one or mode varnode as input, and optionally they can have an output. Only the output varnode can be modified. 

\subsubsection{Address Space}

The address space for p-code is a generalization of RAM. It represents an indexed sequence of bytes in memory that PCode can read and write into it. The address space has an identifying name, a size indicating the number of distinct indexes of memory, and an endianness that specify the encoding used to store integers and multi-byte values into the space address.

A regular processor has a RAM space to model memory accessible via its data bus, a register space to model the processor's registers, and usually a constant space to store all the constants used by PCodes. All the data that PCode handles must be stored into an address space. 
PCode generally uses a temporary address space to store intermediate values when modeling the processor's instructions. 
The implementation of a processor can have as many address spaces as it needs.


\subsubsection{Varnode}
A varnode is a generalization of a register or memory location and has the functionality of handle the data manipulated by PCodes. It is composed of an address space, an offset into the address space, and a size. 
A varnode is a sequence of contiguous bytes that can be treated as a single value. The address and the size identify the varnode. Even if they have no type, some PCode operations can cast them into three types: integer, boolean, or floating-point. 

In the case of integer values, the PCode operation represents the varnode using the endianness linked with the address space.
In the case of floating-point, the operation uses the encoding of the varnodes that depends on its size.
In the case of boolean values, the varnode has a single byte that can be 0 for false, 1 for true.

If the varnode's address space is in the constant space, the varnode is a constant or an immediate value. In this case the size of the varnode is the size or the precision available for the encoding of the constant.



\subsection{Control Flow Graph}
\label{subsec:cfg_ghidra}
We rely on Ghidra's PCode representation to build our dataset for \textit{Control Flow Graph}. Ghidra contains three different scripts for analyzing the flow of a program, and we studied those scripts to understand how Ghidra manages the PCode and their flow. The script iterates all the functions of the given sample and generates a .json file with the extracted data.

Ghidra offers a \texttt{DecompileInterface}, a class that can decompile a function, and that returns an  \texttt{DecompiledResult} object with all the information needed. It is also possible to set different options to the \texttt{DecompileInterface} using the \texttt{DecompileOption} class. The resulting object contains an instance of \texttt{HighFunction}, a high-level abstraction associated with a low-level function made up of assembly instructions. The \texttt{HighFunction} object offers the possibility to iterate over the BasicBlocks of the corresponding function so we can analyze all the blocks and create our graph.

We save the information gathered into a \texttt{.json} file with the following structure:
The \texttt{.json} is composed of an array of basic blocks, each of which has an index, a list of PCodes, and two lists, one containing the indexes of the previous basic blocks and the other one the indexes where the basic block points, i.e., the flow of our function. The PCodes have a field with the associated PCode operation, a list of input varnodes, and a possible output varnode.

The main problem encountered running the script is the decompilation time. Some functions are intricate, and when they come to decompile, Ghidra can take a very long time, even 25 minutes per sample.  Furthermore, the \texttt{DecompileOption} has a field indicating the maximum dimension of the payload of the decompiled function. The default value is 50MB, but for some specific functions, it is still low, and we need to increase it to 500MB to correctly decompile all the functions.

\subsection{Cyclomatic Complexity}


Ghidra offers a class to compute the complexity of a function, \texttt{CyclomaticComplexity}. This class has a method to calculate the cyclomatic complexity of a function by decomposing it into a flow graph using a \texttt{BasicBlockModel}. During the decompilation, we calculate the complexity of each function and stores it into the .json file.


\section{Scikit-learn}
Scikit-learn \cite{scikit-learn} is a Python module integrating a wide range of state-of-the-art machine learning algorithms for medium-scale supervised and unsupervised problems. 

It is open-source, commercially usable, and contains many modern machine learning algorithms for classification, regression, clustering, feature extraction, and optimization.
For this reason, Scikit-learn is often the first tool in a Data Scientists toolkit for machine learning of incoming data sets. 

We choose Scikit-learn as a machine learning library because it is the most used in literature, and it has a great community and support. Moreover, it offers all the models and methods required by our work. 
\section{Jupyter Notebook}

The Jupyter Notebook  \cite{jupyter} is an open-source web application that allows you to create and share documents that contain live code, equations, visualizations and narrative text. Uses include: data cleaning and transformation, numerical simulation, statistical modeling, data visualization, machine learning, and much more.

We choose Jupyter because it excellently fits our need for remote running. In the same server, where we installed Ghidra, we installed Jupyter, and we enabled it to run remotely, with the proper precautions. This setup allows us to remotely work without any problem. 

Machine learning tasks are often time and computationally expensive, so running them on a personal computer would be slower due to limited computer resources, and it would keep the workstation busy.

	\chapter{Features Creation}

When we decided which features could be the most representative for our model, we choose to use only static features. Extracting only static features usually do not require too much time, and there is no need for expensive virtualized systems. In the dynamic analysis, the sample has to be executed in a controlled environment to analyze its behavior. 

Since we want to boost up the work made by Laurenza et al., we choose to continue their path of using only static features, but we added novel features from code analysis.  We are looking for a framework fast and efficient, that can analyze lots of sample without being resource expensive.

At first, we tried to replicate the work done by Caliskan et al., but we found that most of the tools used depend on software no more maintained. Some of those tools do not work as expected, and others were slow in processing files. To simplify as much as possible the process of analyzing executables, we decided to use only Ghidra as software for extracting features.

The following sections cover the blocks of features extracted, explaining the procedure used to accomplish this task.

\section{Dataset file format}
The choice of the format used for storing our dataset was crucial.
We decided to save the feature vector of each sample into a \texttt{.csv } file. Using the md5 we found the related APT campaign in the dAPTaset. Each \texttt{.csv} file contains a column named md5, a column name apt, and \texttt{n} columns corresponding to the features vector of the sample. During the classification phase the md5s are stripped out from the dataset and the apt column is the label of each sample.

To generate the dataset with the feature vectors of all the samples, we tried different approaches. 

The first one was to merge all the \texttt{.csv} files into a single big \texttt{.csv}. Unfortunately, the more features we extracted, the more significant was the dimensionality of our dataset.  When it comes to reading into python, pandas was very slow in both reading and processing the files.

A valid alternative to pandas is \textit{Dask} \cite{dask}, a flexible parallel computing library for analytics, that integrates with \textit{pandas},  \textit{numpy}, and \textit{scikit}. However, the dask-ml package lacks some functionalities for the cross-validation and random forest model. Furthermore, It was still slow in reading bigger files, so we decided to find another solution to speed up the process.

In the end, we decided to store our dataset into a \textit{Hierarchical Data Format} (\textbf{HDF5}) \cite{hdf} designed to store and organize large amounts of data. This format comes with a cost, the files are much bigger, but we drastically improved the speed of reading and processing the dataset.


\section{Disassembled features}

The extraction of disassembled code was an easy and fast task. The documentation provides all the information on how to correctly use the disassembler. We wrote a script that extracts the disassembled code for each function and stores it into a \texttt{.dis} file. The script creates a folder for each sample and stores inside the disassembled code.

From the disassembled code, we extracted 5 kinds of features:
\begin{itemize}
	\item{Line unigrams}
	\item{Disassemble unigrams}
	\item{Disassemble bigrams}
	\item{Instruction only unigrams}
	\item{Instruction only bigrams}
\end{itemize}

First of all, we stripped out all the hexadecimal and numbers, replacing the regex \texttt{"$\backslash$d+"} with the word \texttt{"number"}, and \texttt{"0[xX][0-9a-fA-F]+"} with the word \texttt{"hexadecimal"}. Stripping the numbers and hexadecimal reduced the possibility of overfitting because some numbers may be unique, and that would create a useless feature.


\subsection{Line unigram}
The first block of features is the whole line unigram, we split the disassembled code of each function on the new-line character and then count all the occurrences of different line instructions. We stripped out all the commas because, in the beginning, we saved the dataset to \texttt{.csv} with comma as a separator. For example, the features generated from the function f in Table \ref{tab:function_f} are shown in Table \ref{tab:line_unigrams}

\begin{table}[!htb]
\begin{minipage}{.5\linewidth}
	\centering
	
	\caption{Assembly code of function f}
	\label{tab:function_f}
	
	\medskip
	
	\begin{tabular}{ l } 
		\toprule
		\texttt{push ebx} \\
		\texttt{mov eax, 1}\\
		\texttt{cmp ebx, eax}\\
		\texttt{jle 0xDEADBEEF}\\
		\texttt{add eax, 1}\\
		\texttt{cmp ebx, eax}\\
		\texttt{jle 0xBACADDAC}\\
		\texttt{mov eax, 0x400231BC}\\
		\texttt{call eax}\\
		\texttt{ret}\\
	
		
		\bottomrule
	\end{tabular}
\end{minipage}\hfill
\begin{minipage}{.5\linewidth}
	\centering
	
	\caption{Feature vector of line unigrams}
	\label{tab:line_unigrams}
	
	\medskip
	
	\begin{tabular}{  lr } 
		\toprule
		\makecell{ Feature }  &  Value \\   
		
		\midrule 
		\texttt{push ebx} & 1         \\
		\texttt{mov eax,number} & 1                  \\ 
		\texttt{cmp ebx,eax }   & 2                  \\ 
		\texttt{add ebx, number}     & 2                  \\ 
		\texttt{jle hex }       & 2                  \\
		\texttt{mov eax, hexadecimal} & 1\\ 
		\texttt{call eax}       & 1                  \\
		\texttt{ret} & 1\\
		\bottomrule
	\end{tabular}
\end{minipage}
\end{table}


\subsection{Disassemble unigrams and bigrams}
For this block of features, we split the entire line in instruction, eventual registers, or numbers. Firstly, we split the line on the first space, then if the second half of the string still contains data, we split for all the commas to get the single registers/numbers. The line \texttt{"mov eax, 0x12"} would be split in the following array: \texttt{["mov", "eax", "hexadecimal"]} . As before, we counted the occurrences of every word in the file. 

For the unigram files, we only considered as a feature every word we would obtain after splitting the string. For the bigram files, instead, we considered as a feature the pair of words in the file. 

Furthermore, we added a start token (\texttt{"<s>"}) before the first instruction of every function, and an end token (\texttt{"</s>"}) after the last instruction. We concatenate the first and second element of the bigram with the the string \texttt{"=>"} The features generated from the same disassembled code are presented in Table \ref{tab:dis_uni} and \ref{tab:dis_big}.

\begin{table}[!htb]
	\begin{minipage}{.5\linewidth}
		\centering
		
		\caption{Feature vector of disassemble unigrams}
		\label{tab:dis_uni}
		
		\medskip
		
		\begin{tabular}{  lr } 
			\toprule
			\makecell{ Feature }  &  Value \\   
			
			\midrule 
			\texttt{push} & 1	\\
			\texttt{ebx} & 3\\
			\texttt{mov} & 2\\
			\texttt{eax} & 6\\
			\texttt{number} & 2\\
			\texttt{cmp} & 2\\
			\texttt{jle} & 2\\
			\texttt{hex} & 3\\
			\texttt{add} & 1\\
			\texttt{call} & 1\\
			\texttt{ret} & 1\\
			\bottomrule
		\end{tabular}
	\end{minipage}\hfill
	\begin{minipage}{.5\linewidth}
		\centering
		
		\caption{Feature vector of disassemble bigrams}
		\label{tab:dis_big}
		
		\medskip
		
		\begin{tabular}{  lr } 
			\toprule
			\makecell{ Feature }  &  Value \\   
			
			\midrule 
			\texttt{<s>=>push} & 1	\\
			\texttt{push=>ebx} & 1\\
			\texttt{ebx=>mov} & 1\\
			\texttt{mov=>eax} & 2\\
			\texttt{eax=>num} & 2\\
			\texttt{num=>cmp} & 2\\
			\texttt{cmp=>ebx} & 2\\
			\texttt{ebx=>eax} & 2\\
			\texttt{eax=>jle} & 2\\
			\texttt{jle=>hex} & 2\\
			\texttt{hex=>add} & 1\\
			\texttt{add=>eax} & 1\\
			\texttt{hex=>mov} & 1\\
			\texttt{eax=>hex} & 1\\
			\texttt{hex=>call} & 1\\
			\texttt{call=>hex} & 1\\
			\texttt{hex=>ret} & 1\\
			\texttt{ret=></s>} & 1\\
			\bottomrule
		\end{tabular}
	\end{minipage}
\end{table}


\subsection{Instruction only unigrams and bigrams}
For the last block of features, we decided to study only the frequency of the different instructions in the code, without considering the registry. As before in the bigrams, we added a start and an end token to avoid linking two instructions from different functions. Tables \ref{tab:instr_uni} and \ref{tab:instr_big} show the features generated from the previous example.

\begin{table}[!htb]
	\begin{minipage}{.5\linewidth}
		\centering
		
		\caption{Feature vector of instruction only unigrams}
		\label{tab:instr_uni}
		
		\medskip
		
		\begin{tabular}{  lr } 
			\toprule
			\makecell{ Feature }  &  Value \\   
			
			\midrule 
			\texttt{push} & 1	\\
			\texttt{mov} & 2\\
			\texttt{cmp} & 2\\
			\texttt{jle} & 2\\
			\texttt{add} & 1\\
			\texttt{call} & 1\\
			\texttt{ret} & 1\\
			\bottomrule
		\end{tabular}
	\end{minipage}\hfill
	\begin{minipage}{.5\linewidth}
		\centering
		
		\caption{Feature vector of instruction only bigrams}
		\label{tab:instr_big}
		
		\medskip
		
		\begin{tabular}{  lr } 
			\toprule
			\makecell{ Feature }  &  Value \\   
			
			\midrule 
			\texttt{<s>=>push} & 1	\\
			\texttt{push=>mov} & 1\\
			\texttt{mov=>cmp} & 1\\
			\texttt{cmp=>jle} & 2\\
			\texttt{jle=>add} & 1\\
			\texttt{add=>cmp} & 1\\
			\texttt{jle=>mov} & 1\\
			\texttt{mov=>call} & 1\\
			\texttt{call=>ret} & 1\\
			\texttt{ret=></s>} & 1\\
			\bottomrule
		\end{tabular}
	\end{minipage}
\end{table}

\section{Control Flow Graph features}
For \textit{Control Flow Graph} features we proceed by creating a \texttt{.json} file with all the information gathered from the Decompiler as explained in Subsection \ref{subsec:cfg_ghidra}.

From the CFG files, we extracted three kinds of features:
\begin{itemize}
	\item {Control Flow Graph unigrams complete}
	\item {Control Flow Graph unigrams Pcode only}
	\item {Control Flow Graph bigrams Pcode only}
\end{itemize}

\subsection{Control Flow Graph unigrams complete}
This first set of features contains the unigrams of the complete Pcode representation.
For each sample we extract, from the corresponding \texttt{.json} file, all the pcodes in that function. We consider as a feature every unique combination of pcodes with input and output varnodes.
The key for each feature is the concatenation of the PcodeOP, the input and output varnodes. In particular, we construct the key as follow:
\texttt{PCodeOP\_nodeoutput\#nodeinput*count} of nodes.
The key of the example in \ref{lst:jsonpcode} is \texttt{call\_ram\#const*2}. As before we counted the occurrences of each key and we built our dataset.

\begin{lstlisting}[language=json,firstnumber=1,caption={Example of .json format with PCode},label={lst:jsonpcode}]
{
"pcodes": [
	{
	"code": "CALL",
	"varnode_in": ["ram","const"],
	"count": 2 
	}]
}
\end{lstlisting}



\subsection{Control Flow Graph Pcode only unigrams and bigrams}

These two sets of features contain the unigrams and bigrams of the pcode only. We built the key using only the pcode operator, and then counted the occurrences. For the bigrams, we concatenated as before the key with the string \texttt{"=>"}.

\subsection{Cyclomatic Complexity}

To calculate the Cyclomatic Complexity of a given function, we used the class from Ghidra's API \texttt{CyclomaticComplexity}. The function has a method that returns the complexity of the given function. We saved for every function of each sample its complexity. 

Then, for each sample, we calculated \textit{maximum}, \textit{mean}, and \textit{standard variation} of the complexity of all the functions, and used those as features.

\subsection{Standard Library}

One primary task of reverse engineering binary code is to identify library code. Since what the library code does is often known, it is of no interest to an analyst. Hex-Rays has developed the IDA FLIRT signatures to tackle the problem. 

Function ID is Ghidra's function signature system. Unfortunately, Ghidra has very few Function ID datasets. There is only function identification for the Visual Studio supplied libraries. Ghidra's Function ID allows identifying functions based on hashing the masked function bytes automatically\cite{ghidra_fid}.

We exploit this functionality to determine which of the functions belongs to a standard library and added a boolean field in the \texttt{.json} file, indicating whether that function is or not a standard known function. Then we calculated the number of standard functions in the given sample and use it as a feature.

\section{Rich Header features}
We used the script in paper \cite{dubyk2019sans} to calculate the rich header hash of each sample. Sadly, as pointed out in the paper, not every binary is compiled with the rich header; in fact, only 1669 samples out of 2086 have it. 

The script extracts the \texttt{productID}, the \texttt{productVersion}, and \texttt{productCount}, we concatenated those numbers with a dash \texttt{"-"} to create a key and set 1 if the sample contains the previous key, 0 otherwise.

\section{Total features}
Table \ref{tab:num_feat} shows the number of features of each type, with a total number of \textbf{218012} features.

\begin{table}[!htb]
		\centering
		\caption{Number of features of each type}
		\label{tab:num_feat}
			\begin{tabular}{ll}
				\toprule
				Features type                    & Dimensionality \\
				\midrule
				Disassemble unigrams             & 2476           \\ 
				Disassemble bigrams              & 39697          \\ 
				Disassemble line unigrams        & 25927          \\ 
				Disassemble instruction unigrams & 347            \\ 
				Disassemble instruction bigrams  & 8125           \\ 
				CFG unigrams                     & 13536          \\ 
				CFG bigrams                      & 118852         \\ 
				CFG code unigrams                & 61             \\ 
				CFG code bigrams                 & 2026           \\ 
				CFG complexity                   & 4              \\ 
				Standard library function        & 5577           \\ 
				Rich Header complete             & 1217           \\ 
				Rich Header without count        & 167            \\ 
				\bottomrule
			\end{tabular}
		
\end{table}


	\chapter{Classification and evaluation}
\section{Classification model}
\subsection{Validation}
\section{Features Selection}

	\chapter{Discussion}
\label{ch:disc}

This chapter shows the results obtained during the evaluation and the tests we made to ensure the effectiveness of our models. 

\section{Testing flow?}

We write a function to execute different tests using different classificators and hyperparameters. 

As mentioned in Section \ref{sec:cv}, we used the \texttt{StratifiedKFold} class from \textit{Scikit-learn} to cross-validate our tests. This class has a method \texttt{split} that, given the train and test subsets, it returns a range of indexes to select only a portion of the dataset, as shown in Figure \ref{fig:stratified}.  At each fold, different parts of the dataset are chosen as a test subset, but summing the test subset in all the folds, we would obtain the entire dataset. Therefore we create two arrays where, at each fold, we append the predictions made by the classification model and the ones expected. At the end of the cross-validation algorithm, we use the \textit{Scikit-learn} functions \texttt{classification\_report}, \texttt{confusion\_matrix}, and \texttt{accuracy\_score}, with the complete arrays of predictions and expected values, to obtain the metrics needed to evaluate the model.

We chose $k = 10$ because, as shown in the literature \cite{kohavi1995study}, it's the best value between performances and execution time. 
We set the \texttt{StratifiedKFold} option \texttt{shuffle} to true, so every time the cross-validation algorithm samples different binaries at each fold, avoiding that classification model focuses only on a very good or lousy subset of the dataset. Furthermore, we repeat each test 5 times, and then we average the results obtained.

For \texttt{RandomForest} and \texttt{XGBoost} models, we found that 150 trees are the right compromise between performances and execution time.

\section{Features selection}
In Section \ref{sec:feat_sel}, we presented different methods for feature selection. In this section, we analyze and compare them to find the best subset of features to represent APT malware.

\subsection{Scaling dataset}
The first step for selecting features is to scale the dataset. It is essential because, in our case, we have features with very different scales and contain some outliers. These two aspects can decrease the predictive performance of the classification model. We chose the \texttt{MaxAbsScaler} from \textit{scikit-learn} because it fits our needs. The scaler does not shift or center the data, and it does not destroy the sparsity of the features. For each feature, the algorithm calculates the maximum value and scales all the features in the column, such that the maximum value is equal to 1.0. All the features will be in the range of 0 and 1.0, but they maintain their variance.

\subsection{Remove low variance}
The second step is to visualize the variance of the dataset. We calculated with the std function on axis 0, and we visualize them with matplotlib. Figure \ref{fig:var_all} shows the variance of the entire dataset. As we can see, some features have zero variance, and this means that they are constant over all the samples. Thus they are not useful for classification.

\begin{figure}[!h]
	\centering
	\includegraphics[width=0.6\columnwidth]{variance-all.png}
	\caption{Variance of features in the entire dataset}
	\label{fig:var_all}
\end{figure}

To remove them, we use \texttt{VarianceThreshold} class that removes all low-variance features. Is it possible to set a threshold, if a feature variance is below the threshold, then it is removed. By default, the class removes all zero-variance features. This method removes \textbf{32} features from our dataset.

\subsection{Filter methods}

Filter methods work by selecting the best features based on a ranking function. Scikit-learn offers various scoring functions, such as \textit{chi2, f\_classif, mutual\_info\_classif}. To choose the best $k$ features scikit-learn has two classes:
\begin{itemize}
	\item \texttt{SelectKBest: }removes all but the highest $k$ scoring features.
	\item \texttt{SelectPercentile} removes all but a user-specified highest scoring percentage of features.
\end{itemize} 
We use \texttt{SelectPercentile} to perform some tests with different percentages and compared the results. 
We select respectivly the 25\%, 15\%, and 10\% of the functions \texttt{chi2, fclassif, and mutual\_info} and summarized them in Figures \ref{fig:ranking} and Tables \ref{tab:rank}. The column \textit{time} in Tables \ref{tab:rank} refers to the time elapsed during the feature selection, not during the classification algorithm to test the performances.

\begin{figure}[]
	\centering
	\begin{subfigure}[t]{0.48\textwidth}
		\centering
		\includegraphics[width=\linewidth]{univariance25.png}
		\caption{Ranking functions score with 25\%}\label{fig:ranking25}		
	\end{subfigure}
	\begin{subfigure}[t]{0.48\textwidth}
		\centering
		\includegraphics[width=\linewidth]{univariance15.png}
		\caption{Ranking functions score with 15\%}\label{fig:ranking15}
	\end{subfigure}\\
	\begin{subfigure}[t]{0.48\textwidth}
		\centering
		\includegraphics[width=\linewidth]{univariance10.png}
		\caption{Ranking functions score with 10\%}\label{fig:ranking10}
	\end{subfigure}
	\caption{Comparison of metrics score for different ranking functions with different percentages selected}\label{fig:ranking}
\end{figure}

As we can see from results, the \texttt{f\_classif} is the one who performs the best, even if it takes a little more time than \texttt{chi2} . On the contrary, \texttt{mutual\_info\_fclassif} is the slower, it takes half a day to calculate the mutuals information of the features, and it is also the worst in terms of performance.

However, as we reduce the percentage of features, the performances degrade too. The best percentage is 25\%, but it keeps over 50000 features, which are not good for our purpose. The classification time, instead, is acceptable, it varies from 58s for 25\% features, to 41s for 10\% features. 
We cannot rely only on filter methods because they would not shrink our dataset as we intended. So we tried other methods for feature selection provided by \textit{scikit-learn}.




\begin{table}[]

	\caption{Comparison of metrics score for different ranking functions with different percentages 	\label{tab:rank}}
	\begin{subtable}{\linewidth}
	\centering
	\caption{Summary of ranking functions selecting the 25\% of the best features}
	\label{tab:rank_function}
	\begin{tabular}{llll}
		\toprule
		\textbf{Metrics}  & \textbf{chi2} & \textbf{fclassif }& \textbf{mutual\_info} \\
		\midrule
		\texttt{Time} & 7.60s & 246.05s & 43217.97s\\
		
		\texttt{Accuracy} & 94.90\% &  95.11\% &  94.57\% \\
		\texttt{Precision}  & 94.96\% & 95.14\% &   94.67\%   \\ 
		\texttt{Recall} & 94.90\%  &   95.11\%  & 94.70\% \\ 
		\texttt{F1-score}  &   94.89\%   & 95.10\% &    94.64\%      \\ 
		\bottomrule
	\end{tabular}
	\end{subtable}

	\begin{subtable}{\linewidth}
	\centering
	\caption{Summary of ranking functions selecting the 15\% of the best features}
	\label{tab:rank_function15}
	\begin{tabular}{llll}
		\toprule
		\textbf{Metrics}  & \textbf{chi2} & \textbf{fclassif }& \textbf{mutual\_info} \\
		\midrule
		\texttt{Time} & 7.60s & 246.05s & 43217.97s\\
		
		\texttt{Accuracy} & 94.46\% &  95.06\% &  94.17\% \\
		\texttt{Precision}  & 94.54\% & 95.1\% &   94.27\%   \\ 
		\texttt{Recall} & 94.46\%  &   95.06\%  & 94.24\% \\ 
		\texttt{F1-score}  &   94.42\%   & 95.06\% &    94.17\%      \\ 
		\bottomrule
	\end{tabular}
	\end{subtable}
\begin{subtable}{\linewidth}
	\centering
	\caption{Summary of ranking functions selecting the 10\% of the best features}
	\label{tab:rank_function10}
	\begin{tabular}{llll}
		\toprule
		\textbf{Metrics}  & \textbf{chi2} & \textbf{fclassif }& \textbf{mutual\_info} \\
		\midrule
		\texttt{Time} & 7.60s & 246.05s & 43217.97s\\
		
		\texttt{Accuracy} & 94.19\% &  94.99\% &  93.92\% \\
		\texttt{Precision}  & 94.36\% & 95.05\% &   94.05\%   \\ 
		\texttt{Recall} & 94.19\%  &   94.99\%  & 93.99\% \\ 
		\texttt{F1-score}  &   94.15\%   & 94.98\% &    93.89\%      \\ 
		\bottomrule
	\end{tabular}
\end{subtable}
	
	
\end{table}


\subsection{Embedded Methods}

Embedded methods use models that have build-in feature selection methods. \texttt{RandomForestClassifier} has a field \texttt{feature\_importances} to rank all the features from best to worst. In particular, \textit{scikit-learn} provides a \texttt{SelectFromModel} class, that using a classifier that has a build-in feature importance method, select the best features based on the model. It is possible to set a \texttt{max\_features} parameter to limit the number of selected features.

However, the \texttt{SelectFromModel} algorithm chooses 12781 features in just 18s. Even with \texttt{max\_features} set to 21781 (the 10\% of the entire dataset), the algorithm still chooses only the 12781 features. Comparing the performances with the filter methods presented before, we find out that \texttt{SelectFromModel} is more accurate and the fastest in classification because there are fewer features in the feature vector. Based on those tests we prefer to discard filter methods and keep using \texttt{SelectFromModel}.

\begin{figure}[]
	\centering
	\begin{subfigure}[t]{0.48\textwidth}
		\centering
		\includegraphics[width=\linewidth]{selmodel-time.png}
		\caption{Time comparison}\label{fig:model-time}		
	\end{subfigure}
	\begin{subfigure}[t]{0.48\textwidth}
		\centering
		\includegraphics[width=\linewidth]{selmodel-metrics.png}
		\caption{Metrics comparison}\label{fig:model-metrics}
	\end{subfigure}
	\caption{Comparison of time and metrics score with different number of features selected using \texttt{SelectFromModel}}\label{fig:model}
\end{figure}


We run various tests with different numbers of \texttt{max	features} to compare the results. The tests are shown in Figure \ref{fig:model}. Figure \ref{fig:model-time} shows how time varies related to the number of features selected.

As we can see from Figure \ref{fig:model-metrics} there is no big difference in the tests made between \textit{accuracy, precision,recall, or f1-score}. For example, the maximum difference in accuracy is just 0.0016s. This happens because \texttt{RandomForestClassifier} already performs features selection using \texttt{feature\_importances} during the training phase. Since the time is dependent on the number of features, we choose to balance the performances and the execution time selecting the best 3000 features.

\subsection{Wrapped methods}

As explained in Section \ref{sec:wrapper} wrapper methods are the most computational expensive. \textit{Scikit-learn} has \texttt{RFE} and \texttt{RFECV} classes for recursive feature elimination. 

From \textit{scikit-learn} documentation \cite{rfe} "\textit{The goal of recursive feature elimination (RFE) is to select features by recursively considering smaller and smaller sets of features. First, the estimator is trained on the initial set of features and the importance of each feature is obtained either through a coef\_ attribute or through a feature\_importances\_ attribute. Then, the least important features are pruned from current set of features. That procedure is recursively repeated on the pruned set until the desired number of features to select is eventually reached"}.

The \texttt{RFECV} works the same as \texttt{RFE}, but it uses a cross-validation algorithm to validate the evaluation performance.
The model used is \texttt{RandomForest} and \texttt{StratifiedKFold} as cross-validation technique.

Unfortunately, it is really slow. We tried using the entire dataset, but even after three days of computing, the algorithm didn't stop. We changed strategy, and we tried to use the features calculated from filter methods, but again it took more than three days. In the end, we decide to use the features selected by \texttt{SelctFromModel}. However, \texttt{RFECV} takes a considerable amount of time compared to other feature selection techniques.

In particular, we run tests using both the 3000 and the 1000 features selected before. As expected, the time increases exponentially with the number of features. The 3000 dataset took almost 1 hour to reduce to the optimal number of features, while the 1000 one took only 12 minutes.  Figure \ref{fig:rfecv} shows the decrease in performance as we remove more and more features from the dataset. Figure \ref{fig:rfecv-comparison} shows the comparison of evaluation metrics between the two different datasets.
Comparing the results, we did not notice a big difference in terms of accuracy precision or recall. Even if one algorithm stops at \textbf{2616} features and the other to only \textbf{842}. However, the calculation time was way longer for the first dataset. 

We continue our tests using the output of \texttt{RFECV} for 1000 features, because it was faster to calculate. 
We apply again the \texttt{SelectFromModel} algorithm to lastly reduce the number of features, leaving us with only 305 features.

\begin{figure}[]
	\centering
	\begin{subfigure}[t]{0.48\textwidth}
		\centering
		\includegraphics[width=\linewidth]{rfecv-from3000.png}
		\caption{Initial dataset contains 3000 features}\label{fig:rfecv3000}		
	\end{subfigure}
	\begin{subfigure}[t]{0.48\textwidth}
		\centering
		\includegraphics[width=\linewidth]{rfecv1000.png}
		\caption{Initial dataset contains 1000 features}\label{fig:rfecv1000}
	\end{subfigure}
	\caption{Relation between cross validation scores and number of features.}\label{fig:rfecv}
\end{figure}

\begin{figure}[]
	\centering
	\includegraphics[width=0.6\columnwidth]{rfecv-confronto.png}
	\caption{RFECV applied respectively to a dataset of 3000, and 1000 features.}
	\label{fig:rfecv-comparison}
\end{figure}

We stopped at 305 features, because we did not notice any improvement in terms of classification time. We consider 5.5s an acceptable time for classification. Furthermore, the less features we got, the worst are the performance in evaluation, so we decide to keep that number of features.

Analyzing the type of features we have, we find that almost all the blocks of features still has some feature in the dataset. The Table \ref{tab:num-feat-842} reports the number of features keep per type. The standard library features are removed during features selection, the others are significantly reduced.  

\begin{table}[]
	\caption{Comparison between initial number of features, and number of features after feature selection.}
	\label{tab:num-feat-842}
	\begin{tabular}{lll}
		\toprule
		Features type                    & Number & Initial number\\
		\midrule
		Disassemble Unigrams   &19 & 2476           \\
		Disassemble Bigrams    &62   & 39697          \\
		Disassemble Line Unigrams  &28& 25927          \\
		Disassemble Instruction Unigrams & 9 & 347   \\
		Disassemble Instruction Bigrams  & 19 & 8125    \\
		CFG Unigrams    &26 & 13536   \\
		CFG Bigrams      &316 & 118852  \\
		CFG Code Unigrams  & 14 & 61     \\
		CFG Code Bigrams   & 94 & 2026  \\
		CFG Complexity          & 2 & 3     \\
		Standard Library Function & 0 & 5577  \\
		Rich Header   & 4 & 1217  \\
		\bottomrule	
	\end{tabular}
\end{table}


\section{Model Evaluation}

The following section presents the tests made on the final dataset using two different models: \texttt{RandomForestClassificator} ans \texttt{XGBoost}.


	\chapter{Conclusions and Future Works}
\label{ch:future}

In the last decades the number of malware increased exponentially. Between all those malware, there is a particular type known as Advanded Persitent Threat, that are really dangerous. APT is an advanced adversary which goal is to intrude secretly into comptuter systems of big companies or foreign govenrments, to steal valuable information. We proposed a framework to detect if a sample belongs or not to an APT. The analysts can then concentrate their resources only on samples that could be really threatening. This thesis enrich the static features used in malware triage in \cite{laurenza2017malware}, adding features from code analysis. A classifier is trained with features from disassembled code, control flow graph, cyclomatic complexity and rich header. Even if static features are not sufficient to classify different malware, they are fast and easy to compute. For early identification in a malware APT prioritization framework, they let us reach really promising results. The precision, i.e. the number of false positives, is crucial for this framework. We don't want an analyst to focus on a sample targeted as potential APT when instead is just a random malware. In our thesis, APT samples are identified on average with a precision and accuracy over 95\%, some samples even with a precision ad accuracy of 100\%. Overall, using features selection, we reduced the classification time, and improved a little bit the performance, with respect to the original dataset.  However, the dAPTaset is really small to be considered a complete dataset, and we tested only samples known to be related to some APT.

As future works, first of all, we want to extend this thesis also to non-APT malware. The system should recognize which samples are non-APT and classify the APT membership of the others. 

Secondly, we want to test different models and techniques. We would like to focus on Principle Component Analysis to diminish the dimensionality of our dataset. 

Furthermore, we would like to apply a neural network approach to improve evaluation results. We want to train a network, using the documents extracted above. 
	
	\backmatter
	\cleardoublepage
	\phantomsection % Give this command only if hyperref is loaded
	\addcontentsline{toc}{chapter}{\bibname}
	
	\bibliographystyle{ieeetr}

	
	% Here put the code for the bibliography. You can use BibTeX or
	% the BibLaTeX package or the simple environment thebibliography.
	\bibliography{reference}
\end{document}